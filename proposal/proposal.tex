\documentclass{article}
\usepackage[utf8]{inputenc}
\usepackage{graphicx}
\usepackage{filecontents}
\begin{filecontents}{\jobname.bib}

@online{CHIRTI,
title={Reflectance Transformation Imaging (RTI)l},
  url = {http://culturalheritageimaging.org/Technologies/RTI/},
  urldate = {2018-04-22}
}

@article{earl2011reflectance,
  title={Reflectance transformation imaging systems for ancient documentary artefacts},
  author={Earl, Graeme and Basford, Philip and Bischoff, AS and Bowman, Alan and Crowther, Charles and Hodgson, M and Martinez, K and Isaksen, Leif and Pagi, H and Piquette, KE and others},
  year={2011}
}

@online{bloomberg,
title={UK's oldest hand-written document 'at Roman London dig'},
  url = {https://www.bbc.co.uk/news/uk-england-london-36415563},
  urldate = {2018-04-22}
}

\end{filecontents}

\usepackage{biblatex}
\addbibresource{\jobname.bib}

\title{Reflectance Transformation Imaging Proposal}
\author{Johannes Goslar}

\begin{document}

\maketitle

\section{Background}
Cultural Heritage Imaging describes Reflectance Transformation Imaging as following\cite{CHIRTI}:
\begin{quote}
  RTI is a computational photographic method that captures a subject’s surface shape and color and enables the interactive re-lighting of the subject from any direction. RTI also permits the mathematical enhancement of the subject’s surface shape and color attributes. The enhancement functions of RTI reveal surface information that is not disclosed under direct empirical examination of the physical object.
\end{quote}
To generate the initial data the to-be-analysed object is placed under a
specially crafted dome, which inner hull is sprayed matte black and has single
controllable light sources spread out. The top of the dome has a hole in which a
camera objective fits. Each light source is lighted in order and a picture is
taken with only that light source shining. These images are then transferred to a
computer, where a first program analyses this raw data and repackages it for
later use in RTI viewers. Within these RTI viewers, the user can manipulate the
lightning to reveal previously hidden information.
The University of Oxford is a hub for RTI research, but the main interest was
so far from the Faculty of Classics and the School of Archaeology, which use RTI
processes to analyse archaeological artefacts.\cite{earl2011reflectance} Of
particular note is the current effort to further uncover the meaning of the
so-called `Bloomberg tablets', which are the remains of 405 Roman wax tablets,
dating from 50 AD to 80 AD. Only the wood remains of these tablets, but
scratches in the wood relate to the once written text  A further complication is
the occurred reuse of these tablets, multiple texts can overlap each other.\cite{bloomberg} RTI
can help reveal the information by replacing and automating two human eyes
and a light torch by the means discussed above. The
thesis will aim to integrate the Department of Computer Science with the ongoing
research and finally provide the other stakeholders with better software to
achieve their goals.

\section{Open Questions}
The open questions range from questions specific to the Oxford RTI setup to
general applicable questions, the following areas can likely be forwarded as
part of the thesis:
\begin{itemize}
\item  How can the Oxford RTI setup be further automated? Currently objects are
  placed by hand, the images are transferred by hand, manual preprocessing steps
  are required, disk space is constrained, etc.
\item Can RTI be an interesting part of the `Physically Based Rendering' course?
  The course features some slides already, but a practical could further help
  the understanding. A practical element would work best with an extensible RTI
  core, for which new modules could be written each term the course is held.
\item Can modern software engineering and modern user experience design help the
  researchers uncover information faster? Most RTI software has limited support
  for different operating systems, supported RTI domes, image input formats and
  is generally specific to the circumstances the author was in. None is
  featuring an extensible, modular design, which others could write plugins for.
\item Can integrated collaboration features support the inquires? All current
  RTI software only supports a single-user process. No settings can be shared
  live, no annotations are synchronized automatically, no connectivity is
  provided. An online/cloud based background service integrated into the RTI
  viewer could potentially implement this feature set.
\end{itemize}

These are interesting, but the time frame will likely prohibit the answering:
\begin{itemize}
\item Do precalculated lightning angles improve the resulting images? Currently,
  a glossy billiard ball is placed alongside the analysed object and the
  lightning angle is calculated from the specular reflection on this ball. Given
  automated object placement and centring, the actual angles could be measured
  inside the (custom-build, one-of) dome and then used instead.
\item Can machine learning be used to automatically extract hidden textual
  information from RTI images? Building on the proposed annotation feature to
  create labelled data could machine learning principles be used on regions of
  the raw RTI image to automatically run through different lightning
  configurations and reveal the previously hidden letters?
\end{itemize}

\section{Proposed Method}
The common denominator of all questions above is the need to have an extendable
RTI base, which should provide hooks and plugin options for more specific use
cases. As no RTI base is available, this thesis will develop this basis and an
initial set of plugins. Preliminary discussions ended with the likely technology
stack of:
\begin{itemize}
\item Use of web technologies (ECMAScript, HTML, CSS), so the viewing component can be used from any web
  browser and the analysis component can be run locally inside an `electron'
  shell to have fast access to the raw data (up to 6GB for one object)
\item Use of TypeScript as the main implementation language, as it allows a typeable
  API for plugins, which plain ECMAScript would not allow
\item free software under GPLv3, available on open source platforms like Github
  to allow community collaboration
\end{itemize}
Based on this core the next step will be to achieve parity with the current RTI
analyser and viewer, for which the Oxford Centre for the Study of Ancient
Documents will provide raw datasets and the currently calculated RTI images.
After parity is reached the focus will switch to the implementation of the
plugins proposed above. The given timeframe will unlikely be sufficient for a comprehensive
addition of the machine learning component, as enough labelled
data is unlikely to suddenly available, and likely will have to be done in
further research. The final step will be a user study with the Oxford RTI
hub to evaluate if the implemented plugins helped to uncover more information
faster, the fixing of revealed bugs in the evaluation and the best possible
accommodation of occurring wishes from the end users.

\section{Draft Timetable}
The thesis has to be completed by end of August, which gives a timeframe of 18
weeks, including 1.5 weeks for sickness and 1.5 for other hiccups, leaves 15
weeks, which will be split up accordingly:

\begin{description}
\item[1 - 3] Collection of requirements of all stakeholders, evaluation of
  reusable code from the current viewers, completion of the introduction
  part of the thesis and architecture of the base
\item[4 - 6] Implementation of the new RTI program base, documentation inside
  the thesis, completion of the rollout and distribution pipelines.
\item[\ \ 7 \ \ ] First rollout to Oxford-based users and first user feedback cycle to
  identify potential problems early.
\item[8 - 11] Implementation of the proposed plugins and writing of their
  documentation. Completion of the technical part of the thesis.
\item[12 - 13] Feature freeze, second rollout and public release, followed by
  the user study and evaluation.
\item[14 - 15] Conclusion of user study, future outlook, final additions and proofreading of the thesis.
\end{description}

\printbibliography

\end{document}
